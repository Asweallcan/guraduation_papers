\documentclass[a4paper]{ctexart}
\usepackage{graphicx} 
\usepackage[style=1]{mdframed}
\usepackage{geometry}                % 页边距调整
\geometry{top=3.5cm,bottom=3.5cm,left=3.2cm,right=3.2cm}
\begin{document}
\begin{figure}[t]
\centering
\includegraphics[width=0.5\linewidth]{figure/SchoolName}
\end{figure}
\renewcommand{\arraystretch}{1.6}
\begin{center}
\zihao{3} \textbf{\fangsong 本科生毕业设计(论文)开题报告}
\end{center}
\vskip5cm
\begin{tabular}{cc}
{\Large {\textbf{\fangsong 学\quad 生\quad 姓\quad 名:}} } & \underline{\makebox[9cm][c]{\Large {\textbf{\fangsong 吕世豪}}}}\\
{\Large {\textbf{\fangsong 导师姓名、职称:}} } & \underline{\makebox[9cm][c]{\Large {\textbf{\fangsong 刘传文}}}}\\
{\Large {\textbf{\fangsong 所\quad 属\quad 学\quad 院:}} } & \underline{\makebox[9cm][c]{\Large {\textbf{\fangsong 计算机科学与技术学院}}}}\\
{\Large {\textbf{\fangsong 专\quad 业\quad 班\quad 级:}} } & \underline{\makebox[9cm][c]{\Large {\textbf{\fangsong 软件工程zy1501}}}}\\
{\Large {\textbf{\fangsong 设计(论文)题目:}} } & \underline{\makebox[9cm][c]{\Large {\textbf{\fangsong 心理测评系统的设计与实现}}}}\\
\end{tabular}
\vskip 6cm
\begin{flushright}
{\Large {\textbf{\fangsong \today}}}
\end{flushright}
\pagebreak
\begin{center}
{\Large {\textbf{\fangsong 开题报告填写要求}}}
\end{center}
\begin{enumerate}
\item {\large \fangsong 开题报告应根据教师下发的毕业设计(论文)任务书,在教师的指导下由学生独立撰写。}
\item {\large \fangsong 开题报告内容填写后,应及时打印提交指导教师审阅。}
\item {\large \fangsong “设计的目的及意义”至少800汉字(外语至少500字),“基本内容和技术方案”至少400汉字(外语至少200字)。进度安排应尽可能详细。}
\item {\large \fangsong 指导教师意见:学生的调研是否充分?基本内容和技术方案是否已明确?是否已经具备开始设计(论文)的条件?能否达到预期的目标?是否同意进入设计(论文)阶段?}
\end{enumerate}
\pagebreak

{\large \textbf{\fangsong 撰写内容要求(可加页):}}

{\large \fangsong 
\begin{flushleft}
1.目的及意义(含国内外的研究现状分析)
\end{flushleft}

心理评测是一种比较先进的测试方法,它是指通过一系列手段,将人的某些心理特征数量化,来衡量个体心理因素水平和个体心理差异差异的一种科学测量方法。按评测的内容、对象特点、表现形式、目的、时间、要求等分为若干种类。主要是各机关、企业、组织等用来选拔人才、安置岗位以及对一个人进行诊断、评价、辅助咨询的一种手段,心理测评包含能力测试、人格测试和兴趣测试等。

心理测评在社会生活中的意义:

\begin{enumerate}
	\item 描述:心理评测可以从个体的智力、能力倾向、创造力、人格、心理健康等各方面对个体进行全面的描述,说明个体的心理特性和行为。
	\item 诊断:心理评测可以对同一个人的不同心理特征间的额差异进行比较,从而确定其相对优势和不足,发现行为变化的原因,为决策提供信息。
	\item 预测:心理测评可以确定个体间的差异,并由此来预测不同的个体在将来的活动中可能出现的差别,或推测个体在某个领域未来成功的可能性。
	\item 评价:可以评价个体在学习或者能力上的差异,人格的特点以及相对长处和弱点,评价儿童达到的发展阶段等。
	\item 选拔:心理评测可以客观、全面、科学、定量化地选拔人才。
	\item 安置:了解个体的能力、人格、心理健康等心理特征,从而为因材施教或人尽其才提供依据。
	\item 咨询:心理测评可以为学校的升学就业咨询提供参考,帮助学生了解自己的能力倾向和人格特征,确定最有可能成功的专业或者职业。
\end{enumerate}

心理测评在理论研究中的意义:

\begin{enumerate}
	\item 搜集资料:心理测评是搜集有关心理学资料的一个简单易行而又可靠的方法。
	\item 建立和检验假设:从心里测评的资料中可以发现问题,建立理论假设,并且可通过测量结果进一步来检验这些假设,进而推动心理科学的发展,例如智力结构理论的提出和发展,智力测验就起了重要作用。
\end{enumerate}

心理测评在教育领域中的意义:

\begin{enumerate}
	\item 心理测评是科学教学和学习的良好工具:心理评测可以帮助教师和学生更准确、客观、迅速、全面的了解学生或自己的心理特点,教师便可因材施教,学生也可以选择适合自己心理特点的学习方式。心理测评可以评价教师的教学成功,又可评价学生的学习效果,从而为教育提供反馈信息,一方面可使教师总结自己的经验和教训,改进教学,提高教学质量,另一方面可使学生省思和优化学习方式,提高学习效率。
	\item 心理测评是教育研究和改革的重要手段:迅速大量的搜集学生的信息,从中可以归纳出学生心理发展的普遍规律和特点,从而神话心理学理论研究,进而据此制定出教育目标、内容和方法。此外,任何一种新的教育理论、新的教材、新的教学方法和教育实验的价值,在没有通过测量检验其教育效果之前,都无法科学的评定其价值,而心理测评的结果可以用来评价教育效果,提供客观的精确的依据,进而衡量他们的价值。
\end{enumerate}

国内外现状分析:

心理测评的成千上万的受测者往往来自全国各地,很难进行集中的测试,因此现状是例用计算机互联网的方式进行测评是有利条件。

另外,针对儿童的测评借助计算机的方式实现文字、语音、图画、录像等多种方式的结合。

对受测者需要进行多套经典量表才能全面的评估个体的心理,存在诸多的局限。量表多意味着题量大,答题时间长,受测者经常会出现不良情绪,影响作答效率,宽窄网的研究人员曾对三万人的施测进行研究,发现测评进行20分钟的时候,15\%的受测者会出现疲劳、烦躁等不良情绪,进行30分钟的时候,24\%受测者会出现不良情绪,造成测评的效率低。多套量表的设定会出现维度重复甚至是不需要评测的维度,造成评测资源的重复和浪费,而且还会影响后期数据的录入和清理,浪费大量的时间和人工成本。

国内目前的心理咨询师培养要求和国际上的要求距离甚远,这有待国家某些制度的完善。心理学研究的领域非常的广泛,APA(美国心理协会)就有53个学科分支,每个分支都有其广阔的发展前景。从偏于基础性的实验心理学、认知心理学、生理心理学,到偏于应用性的教育心理学、社会心理学、军事心理学、管理心理学等。

}

{\large \fangsong 
\begin{flushleft}
2.	研究(设计)的基本内容、目标、拟采用的技术方案及措施
\end{flushleft}

心理学中的心理测评是很重要的,可以综合评价人的各个素质,可以对自己有一个全面的认识。现在心理测评正发挥着越来越重要的作用。本次毕业设计要求在学习和掌握js等前端框架和Python 等后端技术的基础上,设计并实现一个心理测评系统。该系统主要有普通用户和管理员两种角色。其中:

\begin{enumerate}
	\item  普通用户可以通过本系统完成用户注册、心理测评、报告生成、数据分析等功能;用户可自定义个人账号及部分相关模块(包括个人信息设置、个人心理档案管理)。
	\item 管理员用户除具备普通用户功能外,还具有系统管理、批量导入用户信息、管理用户、导入文件及新的心理测评量表、导出测评报告、所有用户的档案管理等功能。
\end{enumerate}


拟采用技术方案: 前后端分离,因为非服务端渲染,登录状态使用cookie进行识别和储存。

前端使用React 16.x框架,使用Javascript的超集Typescript进行开发,提供类型检查,接口,泛型等编译语言的特性,提高开发稳定性。前端状态管理使用Redux,搭配React-observable,React-redux,Rxjs进行Ajax的流程管理,减少前端Dom结构的重复渲染。开发阶段的代码检查使用Eslint配合Tslint。开发阶段代码的编译和打包使用Webpack 4.x。

后端使用Python Flask框架,数据库使用Mysql,搭配Flask-Sqlalchemy进行数据库对象ORM化、Flask-mail发送邮件、Flask-Migrate进行数据库迁移和备份、api接口使用requests包模拟客户端进行测试,使用itsdangerous提供加密token,使用werkzeug.security进行用户的密码加密。

措施:难点主要在于试卷的解析和答案的得分计算上面,所以对于管理员上传的试卷表格有很高的格式要求,需要开发者提供模板。

}

\begin{flushleft}
3.	进度安排
\end{flushleft}

2019/01/19—2019/02/15:确定选题,查阅文献。

2019/02/16—2019/02/28:外文翻译和撰写开题报告。

2019/03/01—2019/04/01:系统架构、程序设计与开发。

2019/04/02—2019/04/30:系统测试与完善。

2019/05/01—2019/05/25:撰写及修改毕业论文。

2019/05/26—2019/06/05:准备答辩。

\begin{flushleft}
4.阅读的参考文献不少于15篇(其中近五年外文文献不少于3篇)
\end{flushleft}

[1] 未来科技. jQuery 实战从入门到精通 [J]. 水利水电出版社, 2017. 

[2] 叶维忠. Python 编程从入门到精通 [J]. 人民邮电出版社, 2017. 

[3] 杜文洁. 高等学校毕业设计(论文)指导教程——电子信息类专业 [J]. 水利水电出 版社, 2015. 

[4] 刘一奇. React 与 Redux 开发实例精解 [J], 2016. 

[5] 程墨. 深入浅出 React 和 Redux[J]. 机械工业出版社, 2017. 

[6] 程墨. 深入浅出 RxJS[J]. 机械工业出版社, 2018. 

[7] 张静. 慧眼识人——心理学评测在人才管理中的应用 [J]. 辽宁科技, 2016.

[8] 吴浩麟. 深入浅出 webpack[J]. 电子工业出版社, 2018. 

[9] Crockford. JavaScript:The Good Parts[J]. Southeast University Press, 2008. 

[10] BANKS A, PORCELLO E. Python 编程从入门到精通 [J]. China Electric Power Press, 2017. 

[11] JAWORSKI M, ZIADE T. Expert Python Programming[J]. PacktPublishing, 2016. 

[12] RAMALHO L. Fluent Python[J]. Posts and Telecommunications Press, 2017. 

[13] GORELICK M, OZSVALD I. High Performance Python[J]. Posts and Telecommunications Press, 2017. 

[14] Miguel Greenberg. Flask Web Development[J]. Posts and Telecommunications Press, 2018. 

[15] Gustave Le Bon. The crowd: a study of the popular mind[J]. Guangxi Normal University Press, 2011.

\begin{flushleft}
5.指导教师意见

\end{flushleft}
\vskip7.8cm
\begin{flushright}
指导教师(签名):\quad \quad \quad 

\today
\end{flushright}                  

\bibliographystyle{gbt7714-2005}
\end{document}