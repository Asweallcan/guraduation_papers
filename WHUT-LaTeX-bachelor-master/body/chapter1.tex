\pagenumbering{arabic}
\section{心理评测的意义}

心理评测是一种比较先进的测试方法,它是指通过一系列手段,将人的某些心理特征数量化,来衡量个体心理因素水平和个体心理差异差异的一种科学测量方法。按评测的内容、对象特点、表现形式、目的、时间、要求等分为若干种类。主要是各机关、企业、组织等用来选拔人才、安置岗位以及对一个人进行诊断、评价、辅助咨询的一种手段,心理测评包含能力测试、人格测试和兴趣测试等。

\subsection{心理测评在社会生活中的意义}

\begin{enumerate}
\item 描述:心理评测可以从个体的智力、能力倾向、创造力、人格、心理健康等各方面对个体进行全面的描述,说明个体的心理特性和行为。
\item 诊断:心理评测可以对同一个人的不同心理特征间的额差异进行比较,从而确定其相对优势和不足,发现行为变化的原因,为决策提供信息。
\item 预测:心理测评可以确定个体间的差异,并由此来预测不同的个体在将来的活动中可能出现的差别,或推测个体在某个领域未来成功的可能性。
\item 评价:可以评价个体在学习或者能力上的差异,人格的特点以及相对长处和弱点,评价儿童达到的发展阶段等。
\item 选拔:心理评测可以客观、全面、科学、定量化地选拔人才。
\item 安置:了解个体的能力、人格、心理健康等心理特征,从而为因材施教或人尽其才提供依据。
\item 咨询:心理测评可以为学校的升学就业咨询提供参考,帮助学生了解自己的能力倾向和人格特征,确定最有可能成功的专业或者职业。
\end{enumerate}

\subsection{心理测评在理论研究中的意义}

\begin{enumerate}
\item 搜集资料:心理测评是搜集有关心理学资料的一个简单易行而又可靠的方法。
\item 建立和检验假设:从心里测评的资料中可以发现问题,建立理论假设,并且可通过测量结果进一步来检验这些假设,进而推动心理科学的发展,例如智力结构理论的提出和发展,智力测验就起了重要作用。
\end{enumerate}

\subsection{心理测评在教育领域中的意义}

\begin{enumerate}
\item 心理测评是科学教学和学习的良好工具:心理评测可以帮助教师和学生更准确、客观、迅速、全面的了解学生或自己的心理特点,教师便可因材施教,学生也可以选择适合自己心理特点的学习方式。心理测评可以评价教师的教学成功,又可评价学生的学习效果,从而为教育提供反馈信息,一方面可使教师总结自己的经验和教训,改进教学,提高教学质量,另一方面可使学生省思和优化学习方式,提高学习效率。
\item 心理测评是教育研究和改革的重要手段:迅速大量的搜集学生的信息,从中可以归纳出学生心理发展的普遍规律和特点,从而神话心理学理论研究,进而据此制定出教育目标、内容和方法。此外,任何一种新的教育理论、新的教材、新的教学方法和教育实验的价值,在没有通过测量检验其教育效果之前,都无法科学的评定其价值,而心理测评的结果可以用来评价教育效果,提供客观的精确的依据,进而衡量他们的价值。
\end{enumerate}

\subsection{国内外心理评测现状分析}

心理测评的成千上万的受测者往往来自全国各地,很难进行集中的测试,因此现状是例用计算机互联网的方式进行测评是有利条件。

另外,针对儿童的测评借助计算机的方式实现文字、语音、图画、录像等多种方式的结合。

对受测者需要进行多套经典量表才能全面的评估个体的心理,存在诸多的局限。量表多意味着题量大,答题时间长,受测者经常会出现不良情绪,影响作答效率,宽窄网的研究人员曾对三万人的施测进行研究,发现测评进行20分钟的时候,15\%的受测者会出现疲劳、烦躁等不良情绪,进行30分钟的时候,24\%受测者会出现不良情绪,造成测评的效率低。多套量表的设定会出现维度重复甚至是不需要评测的维度,造成评测资源的重复和浪费,而且还会影响后期数据的录入和清理,浪费大量的时间和人工成本。

国内目前的心理咨询师培养要求和国际上的要求距离甚远,这有待国家某些制度的完善。心理学研究的领域非常的广泛,APA(美国心理协会)就有53个学科分支,每个分支都有其广阔的发展前景。从偏于基础性的实验心理学、认知心理学、生理心理学,到偏于应用性的教育心理学、社会心理学、军事心理学、管理心理学等。

\subsection{心理评测系统}

市面上已经有很多知名的评测系统,当然也不乏心理评测系统。一个基本的评测系统要能进行用户的注册登录,做题,得到自己的结果,同时管理人员可以进行试卷的管理和上传,用户的权限设置,用户的批量上传,以及对成绩的筛选。好的心理评测系统可以做到后面遇到的题目和之前选择的答案相关,以及可以通过音频和图片来进行问题和答案的阐述。此心理评测系统具有基本的用户功能和管理功能,可以通过音频来进行问题和答案的阐述。
























