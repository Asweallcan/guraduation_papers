\pagenumbering{arabic}
\section{需求分析}

心理评测是一种比较先进的测试方法,它是指通过一系列手段,将人的某些心理特征线性化,衡量个体心理因素水平和个体心理差异的一种科学测量方法。按评测的内容、对象特点、表现形式、目的、时间、要求等分为若干种类。主要是各机关、企业、组织等用来选拔人才、安置岗位以及对一个人进行诊断、评价、辅助咨询的一种手段,心理测评包含能力测试、人格测试和兴趣测试等。

\subsection{心理测评的意义}

\subsubsection{心理评测在社会中的意义}

(1) 描述:心理评测可以从个体的智力、能力倾向、创造力、人格、心理健康等各方面对个体进行全面的描述,说明个体的心理特性和行为。

(2) 诊断:心理评测可以对同一个人的不同心理特征间的额差异进行比较,从而确定其相对优势和不足,发现行为变化的原因,为决策提供信息。 

(3) 预测:心理测评可以确定个体间的差异,并由此来预测不同的个体在将来的活动中可能出现的差别,或推测个体在某个领域未来成功的可能性。 

(4) 评价:可以评价个体在学习或者能力上的差异,人格的特点以及相对长处和弱点,评价儿童达到的发展阶段等。 

(5) 选拔:心理评测可以客观、全面、科学、定量化地选拔人才。 

(6) 安置:了解个体的能力、人格、心理健康等心理特征,从而为因材施教或人尽其才提供依据。 

(7) 咨询:心理测评可以为学校的升学就业咨询提供参考,帮助学生了解自己的能力倾向和人格特征,确定最有可能成功的专业或者职业。 

\subsubsection{心理测评在理论研究中的意义}

(1) 搜集资料:心理测评是搜集有关心理学资料的一个简单易行而又可靠的方法。

(2) 建立和检验假设:从心里测评的资料中可以发现问题,建立理论假设,并且可通过测量结果进一步来检验这些假设,进而推动心理科学的发展,例如智力结构理论的提出和发展,智力测验就起了重要作用。

\subsubsection{心理测评在教育领域中的意义}

(1) 心理测评是教学和学习的优良工具:通过心理评测教师和学生可以了解自己的心理状态,教师对学生的心里评测结果可以进行分析,改善自己的教学方法,学生能从自己的心里评测报告改善自己的生活习惯、思考方式和学习方式。老师可以做到因材施教,学生可以找到自己适合的学习方式。

(2) 心理测评创新教育的重要手段:收集学生们的心里评测结果,可以对当前教育形式做出判断,从而改善教育结构和方针,修正未来教育的路线,改进整个社会教育的结构。

\subsection{国内外心理评测现状分析}

国外发达国家对心理健康问题的研究起步早,发展力度大,早已形成了比较健全和成熟的心理健康服务体制。以美国为例,高校心理辅导人员配比较为充足,并且从事心理咨询工作的人员呢必须达到由APA和NASP两个专业组织制定规定的专业水平才有资格从业。并且由于在较早的年龄阶段就开始关注公民的心理健康,所以公民可以把接受心理治疗与身体治疗同等看待,这都使得线上面对面式的心理综合测评较为容易实施。

但是我国对心理问题的关注,起步晚,发展慢,国家投入与学科支撑还不足够,也没有充足的相关从业人员,各地的心理健康服务机构也并不完善。需要进行心理测评的用户分布广散,很难进行集中的测试。更为重要的是,由于目前我国在此方面的宣传与指导缺乏,许多人不愿意接受直接的心理咨询与辅导,这就使得很难有机会进行线下的心理测评。基于这些现实情况,使用计算机线上评测能够减少地域上带来的不便和隐私问题,保护用户的隐私可以让用户在做评测时候更加的真实,让测量结果更加准确。

常用的心里评测需要进行多套经典量表才能全面评估个体心理状况,存在诸多的局限。当题量大的时候,根据不完整调查,15\% 的受测者会出现疲劳、烦躁等不良情绪,进行 30 分钟的时候,24\% 受测者会出现不良情绪,造成测评的效率低。多套量表的设定还会出现维度重复甚至产生不需要测评的维度的问题,造成测评资源的重复和浪费。这也会影响后期数据的录入和清理,浪费大量时间和人力成本。

\subsection{需求分析}

\subsubsection{功能性需求}

(1) 用户进行注册登录,系统保存用户信息并对密码进行加密存储。

(2) 用户可以查看并修改自己的个人信息。

(3) 用户进行心理测评,即可完成系统所要求完成的题目。

(4) 用户可以在完成测评后可以获取到分数和图表形式的测评结果。

(5) 管理员对试卷进行相关操作,例如查看、上传、管理。

(6) 管理员可以对用户进行批量管理与权限设置。

(7) 管理员可以根据成绩进行筛选分类,按照某些标准进行划分。

\subsubsection{非功能性需求}

(1) 页面响应迅速,不会出现白屏等情况。

(2) 对于难以表述的题目,可选择使用图片、音频等方式进行辅助。

(3) 保护用户隐私,避免个人隐私信息的泄露。

(4) 用户可在疲倦时暂停答题,系统将对已答部分进行保存,下次进入时可选择继续完成。

(5) 整体设计友好温和,尽量不给受测者压力,保障测评结果尽量少受外界因素的干扰。

\subsection{系统目标}

目前网络上较为流行的各类网页式心理测试又存在不系统不完整、信息无法储存、无法管理、无法对比查看、试卷来源不可靠、风险性较高、测试结果不详细没有指导意义等问题。该类测试随机性较大,传播速度快,传播范围广,但是受测者经过一次测试后很难重复测试持续跟踪,也就是说测试结果很难真正的给予受测者实质性的评判依据和进一步的指导,仅仅是停留在非常表面的随手答题阶段。

基于以上客观现实和待解决问题,本次毕设希望做到,每个人可以根据我们系统的分析结果了解当前自己的心理状态,明确自己的目标,改进自己的不足,使得心理测评系统真正的产生实质性功效。