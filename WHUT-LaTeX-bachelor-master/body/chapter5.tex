\section{结论与展望}

\subsection{系统完成}

心理评测系统完成的功能性需求:

(1) 用户注册、登陆、邮箱验证、找回密码

(2) 管理员管理试卷、管理用户、筛选成绩

(3) 用户参加评测、查看评测结果,用户根据结果了解当下的心理状态,同时可以明确以后在生活中需要做哪些调整

心理评测系统完成的非功能性需求:

(1) 页面响应迅速,不会出现白屏等情况。

(2) 对于难以表述的题目,使用图片进行辅助。

(3) 用户可在疲倦时暂停答题,系统将对已答部分进行保存,下次进入时可选择继续完成。整体设计友好温和,尽量不给受测者压力,保障测评结果尽量少受外界因素的干扰

\subsection{未来展望}

(1) 当前心里评测系统对于试卷的管理还不够完善,只能通过上传excel的表格的形式,调整题目的时候只能删除当前的试卷后再进行上传。未来希望做到能够在页面上新增题目和修改题目的信息。

(2) 权限层级不够多,只设有普通用户和管理员两个权限。未来希望加入权限管理系统的权限,对不同的页面都有一个权限列表而不是单纯的区分角色。

(3) 筛选成绩得到的用户列表暂时没有什么用处。未来希望根据成绩筛选出用户以后能对用户进行一些操作,比如发送邮箱类似一样的通知,以邮件的方式通知用户报告或者一些结论性的东西。

(4) 题目之间没有关联性。未来希望题目是动态化的,后面的题目都是根据前面已做的题目计算得来的。

(5) 数据分析还很浅显,没有更加复杂层级的数据分析。未来希望更加的完善数据分析的维度和算法的复杂度。
