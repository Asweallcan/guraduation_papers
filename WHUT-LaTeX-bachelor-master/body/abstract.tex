\section*{\zihao{2} \centering 摘要}

\vskip0.5cm
心理评测是一种比较先进的测试方法,它是指通过一系列手段,将人的某些心理特征数量化,来衡量个体心理因素水平和个体心理差异差异的一种科学测量方法。按评测的内容、对象特点、表现形式、目的、时间、要求等分为若干种类。主要是各机关、企业、组织等用来选拔人才、安置岗位以及对一个人进行诊断、评价、辅助咨询的一种手段,心理测评包含能力测试、人格测试和兴趣测试等。

本论文实现了一个B/S架构的心理评测系统,基于Python的flask框架,React框架和MySQL数据库,实现了用户的注册登录、账号激活、找回密码、管理员管理试卷、管理员管理用户、计算分数、分析答案得出结论的功能。

使用系统可以进行心理评测并得出结论,每份试卷针对性的让使用者了解自己对应指标的分数,更加了解自己对应方面的优点和缺点,同时也使管理员能够根据成绩进行筛选用户,了解大众的心理趋势。


\textbf{关键词:}  心理评测,结果分析,档案管理,目标筛选
\addcontentsline{toc}{section}{摘要}

\clearpage
\section*{\zihao{2} \centering \textbf{Abstract} }
   %用了Times New Roman字体来美化观感

Psychological assessment is a relatively advanced testing method. It refers to a scientific measurement method to measure the level of individual psychological factors and individual psychological differences by quantifying some psychological characteristics of human beings through a series of means. According to the content, object characteristics, manifestation, purpose, time and requirements of the evaluation, it can be divided into several categories. It is mainly a means used by various organs, enterprises and organizations to select talents, place posts, diagnose, evaluate and assist a person in counseling. Psychological assessment includes ability test, personality test and interest test. Generally speaking, psychological assessment has great significance in education, society and work. This system completes the basic function of psychometric assessment.

This paper implements a psychological evaluation system based on B/S architecture. Based on Python flask framework, React framework and MySQL database, the functions of user registration, account activation, password retrieval, administrator management test paper, administrator management user, calculating scores, analyzing answers and drawing conclusions are realized.

The system can be used to carry out psychological evaluation and draw conclusions. Each test paper aims to let users know their corresponding index scores, better understand the advantages and disadvantages of their corresponding aspects. At the same time, administrators can screen users according to their scores and understand the psychological trends of the public.

\textbf{Key Words:} Psychological assessment, Result analysis, File management, Target selection \\
\addcontentsline{toc}{section}{Abstract}




